\documentclass[10pt]{article}
\usepackage{amsmath}
\usepackage{graphicx}
\usepackage{color}
\usepackage{bm}
%\usepackage{epstopdf}
\usepackage{fullpage}
\usepackage{caption}
\usepackage{subcaption}

\graphicspath{{../plots/}}

\begin{document}
\section*{Question 1}
The first order differential equation for the position $x(t)$ and velocity $v(t)$ are given by the following equations:
\begin{align}
\frac{dx(t)}{dt} &= v(t), \\
\frac{dv(t)}{dt} &= \frac{F(x(t))}{m}.
\end{align}
%
Now, according to the \textbf{forward Euler} method the L.H.S of the above equations can be expressed  as:
\begin{align*}
\frac{x(t + \Delta t) - x(t)}{\Delta t} &= v(t), \\
\frac{v(t + \Delta t) - v(t)}{\Delta t} &= \frac{F(x(t))}{m}.
\end{align*}
%
Using the notation $t=i\Delta t$ where $\Delta t$ is the length of the interval between two consecutive time steps, and, $i$ is an integer $1\leq i \leq \mathrm{Maxstap}$. Substituting for $t$ gives the following expression:png
%
\begin{align*}
x[i+1] &= x[i] + v[i]*\Delta t, \\
v[i+1] &= v[i] + \frac{F(x[i])}{m},
\end{align*}
%
where $x(t+\Delta t) = x((i+1)\Delta t)$ which is written as $x[i+1]$ in the array form. $x(i\Delta t)$ in actual time would mean $x[i]$ in the array form.

\section*{Question 2}
For the case of a simple harmonic oscillator we have two parameters: spring constant $k [kg/s^2]$ and the mass of the block/particle attached to the oscillator $m [kg]$. Therefore, the natural frequency $\omega_{0}$ and time period $t_{p}$ of the spring is given by:

\begin{align}
\omega_{0} &= \sqrt{\frac{k}{m}},\\
t_{p} &= \frac{2\pi}{\omega_{0}}=2\pi \sqrt{\frac{m}{k}}.
\end{align}

Therefore, in our simulation, we would like to observe the oscillator over several periods, and, within each period, we would like to resolve the motion with a reasonable resolution. Hence, based on this argument, we can use the following estimates:

\begin{align*}
T &\approx \mathcal{O}(t_{p}) \\
\Delta t &\approx \mathcal{O}(t_{p}/100) \\
\end{align*}

Obviously, it is beneficial if we can find a higher $\Delta t$. This would imply a reduction in the computational time, but, we might incur errors. Hence, a right balance needs to be achieved. 

\section*{Question 3}
\paragraph{Energy conservation}: The harmonic oscillator is a system with only conservative forces. Hence, the system \textbf{should} conserve energy in time. Therefore, one of the criterion to check the correctness of the simulation is to check if the total energy at any point of time is equal to the energy at time $t=0$. The total energy $E$ as a function of time $t$ is given by:
%
\begin{align*}
E_{tot}(t) &= E_{\mathrm{kin}} + E_{\mathrm{kin}},\\
           &= \frac{1}{2}mv(t)^2 + \frac{1}{2}k(x(t)-x_{0})^2,
\end{align*}
%
where $v(t)$ is the instantaneous velocity, $x(t)$ is the instantaneous position and $x_{0}$ is the position at which the mass experiences 0 force.
 
\paragraph{Analytical solution}: Additionally, an analytical solution for the position and velocity can be derived for this system. Therefore, obtaining the position and velocity as a function of time and comparing it to the analytical expressions can help us check the correctness of the simulation. The analytical solution for the position and velocity is given by the following expressions:
%
\begin{align}
x(t) &= \frac{v_{0}}{\omega_{0}}\sin(\omega_{0}t) + x_{0}\cos(\omega_{0}t),\\
v(t) &= v_{0}\cos(\omega_{0}t) - x_{0}\omega\sin(\omega_{0}t),
\label{eqn:harmonic_oscillator_analytical_soln}
\end{align} 
%
where $x_{0}, v_{0}$ are the initial conditions for the harmonic oscillator. Hence, the above equations represent the general solution for the position and velocity, respectively. 

\paragraph{Phase space}: Finally, another check is that to obtain the trajectory of the harmonic oscillator in the phase space. For a harmonic oscillator, without any damping or forcing, the trajectory in phase space should follow a ellipse.\footnote{ This is just another manifestation of the energy conservation principle.}. The equation of the ellipse obtained from Eqn. \eqref{eqn:harmonic_oscillator_analytical_soln} after squaring, adding and rearranging the resulting terms is:

\begin{equation}
\frac{x^2(t)}{a^2} + \frac{v^2(t)}{b^2} = 1,
\label{eqn:phase_space_eqn}
\end{equation}

where,
\begin{align*}
a^2 =& \Big(\frac{v_{0}^2 + x_{0}^2 \omega_{0}^2}{\omega_{0}^2}\Big),\\
b^2 =& v_{0}^2 + x_{0}^2 \omega_{0}^2.
\end{align*}

\section*{Question 4}
The oscillator has been simulated in MATLAB.

\section*{Question 5}
As stated in \textbf{Question 3} we test for the following criterion:
\begin{itemize}
\item Energy conservation
\item Analytical expression for $x(t)$ vs $t$
\item Phase space of $x(t)$ vs $v(t)$.
\end{itemize}
\paragraph{Energy Conservation}: We compare the energy $E_{tot}(t)$ over several periods of the oscillator. As argued in \textbf{Question 3}, the system should not lose energy since only conservative forces are present in the system. 

\begin{figure}[!htb]
\centering
\input{../plots/q5_euler_dt_2e-4_energy_vs_time.tex}
\caption{The kinetic, potential and the total energy are plotted as a function of normalized time. The normalized time represents the number of periods executed by the harmonic oscillator. The parameters used for this plot are: $m=2\ \mathrm{kg}$, $k=5\ \mathrm{N/m}$, $dt=2\times 10^{-4}t_{p}\ \mathrm{s}$. The forward Euler scheme is used for integrating the equations of motion.}
\end{figure}

\paragraph{Analytical expression}: We compare the position $x(t)$ over several periods of the oscillator. As argued in \textbf{Question 3}, the system should exactly follow the analytical expression for the position.

\begin{figure}[!htb]
\centering
\input{../plots/q5_euler_dt_2e-4_position_vs_time.tex}
\caption{The position of the harmonic oscillator as a function of normalized time is compared against the analytical expression. The normalized time represents the number of periods executed by the harmonic oscillator. The parameters used for this plot are: $m=2\ \mathrm{kg}$, $k=5\ \mathrm{N/m}$, $dt=2\times 10^{-4}t_{p}\ \mathrm{s}$.The forward Euler scheme is used for integrating the equations of motion.}
\end{figure}

\paragraph{Phase space}: The phase space trajectory is plotted for the harmonic oscillator. As derived in Eqn \eqref{eqn:phase_space_eqn}, the points in the phase space must, strictly, lie on the ellipse for a given initial velocity and position. 

\begin{figure}[!htb]
\centering
\input{../plots/q5_euler_dt_2e-4_phase_space.tex}
\caption{The phase space trajectory of the harmonic oscillator is plotted. The $y$-axis contains the position $x(t)$ and the $x$-axis contains the velocity $v(t)$. The parameters used for this plot are: $m=2\ \mathrm{kg}$, $k=5\ \mathrm{N/m}$, $dt=2\times 10^{-4}t_{p}\ \mathrm{s}$.The forward Euler scheme is used for integrating the equations of motion.}
\end{figure}

\section*{Question 6}
In the previous section a small value of $dt$ was used for integrating the equations of motion. However, using larger time steps decreases the accuracy of the forward Euler method. This question focuses on showing this. 

\begin{figure}[!htb]
\centering
\input{../plots/q6_euler_compare_dt_energy_vs_time.tex}
\caption{The total energy of the oscillator is plotted as a function of normalized time for different time steps. The $y$-axis contains the position $E_{tot}(t)$ and the $x$-axis contains the velocity $\tilde{t}$. The parameters used for this plot are: $m=2\ \mathrm{kg}$, $k=5\ \mathrm{N/m}$, $dt=2\times 10^{-4}t_{p}\ \mathrm{s}$.The forward Euler scheme is used for integrating the equations of motion.}
\label{fig:euler_dt_compare}
\end{figure}

\begin{itemize}
\item As can be seen in Fig. \ref{fig:euler_dt_compare}, the \emph{larger} the time step used for the forward Euler, the greater the error in predicting the total energy. As the time step becomes 
\item Another interesting observation is that the forward Euler is an energy accumulating scheme, rather than energy-losing scheme. The derivation for this is shown in the appendix.
\end{itemize}

\section*{Question 7}
The influence of the time step on the Verlet leap-frog scheme is plotted in Fig. \ref{fig:verletLF_dt_compare}. 

\begin{figure}[!htb]
\centering
\input{../plots/q7_verletLF_compare_dt_energy_vs_time.tex}
\caption{The total energy of the oscillator is plotted as a function of normalized time for different time steps. The $y$-axis contains the position $E_{tot}(t)$ and the $x$-axis contains the velocity $\tilde{t}$. The parameters used for this plot are: $m=2\ \mathrm{kg}$, $k=5\ \mathrm{N/m}$, $dt=2\times 10^{-4}t_{p}\ \mathrm{s}$.The Verlet leap-frog scheme is used to integrate the equations of motion.}
\label{fig:verletLF_dt_compare}
\end{figure}

It can be seen that the for large timesteps the scheme does not lose energy. Further, on inspecting the scheme for $dt=2\mathrm{e-}2$ we observe that the oscillator does not lose energy. This is in stark contrast to the Euler scheme which loses energy on using a larger time step. 

Hence, it can be concluded that the performance of the integrator is less sensitive, when compared to the Euler scheme, to the magnitude of time step. On further decreasing the resolution (increasing the magnitude of $dt$) the magnitude of the oscillations of the energy increase. This is shown in Fig. \ref{fig:verletLF_dt_compare_oscillations}.

\iffalse
\begin{figure}[!htb]
\centering
\input{../plots/q7_verletLF_dt_oscillation_energy_vs_time.png}
\caption{The total energy of the oscillator is plotted as a function of normalized time for different time steps. The $y$-axis contains the position $E_{tot}(t)$ and the $x$-axis contains the velocity $\tilde{t}$. The parameters used for this plot are: $m=2\ \mathrm{kg}$, $k=5\ \mathrm{N/m}$, $dt=2\times 10^{-4}t_{p}\ \mathrm{s}$.The Verlet leap-frog scheme is used to integrate the equations of motion.}
\label{fig:verletLF_dt_compare_oscillations}
\end{figure}
\fi

\section*{Question 8}
As mentioned, the position and calculated trajectory remains unaffected with the different interpretation of the velocity. Hence, the potential energy remains unchanged. However, the kinetic energy and also the total energy will depend on the method of calculation of the velocity at integral time steps. The difference in the total energy is shown in Fig. \ref{fig:verletLF_dt_compare_intermediate_velocities}. 

\begin{figure}[!htb]
\centering
\input{../plots/q8_verletLF_compare_dt_energy_vs_time_intermediate_velocities.tex}
\caption{The total energy of the oscillator is plotted as a function of normalized time for different velocity interpretations of the Verlet leap-frog scheme. The $y$-axis contains the position $E_{tot}(t)$ and the $x$-axis contains the velocity $\tilde{t}$. The parameters used for this plot are: $m=2\ \mathrm{kg}$, $k=5\ \mathrm{N/m}$, $dt=2\times 10^{-4}t_{p}\ \mathrm{s}$.}
\label{fig:verletLF_dt_compare_intermediate_velocities}
\end{figure}

As observed, the oscillations in the energy calculation are significantly suppressed when using the average of the intermediate velocities $w$ instead of directly adopting the value of $w$. It should be noted that the oscillations in the case of the average intermediate velocities are also present, however, the maximum amplitude of the oscillation is of the order of $10^{-6}$. 

\section*{Question 9}
The equations for $x((i+1)\Delta t)$ and $x((i-1)\Delta t)$ are given by:
%
\begin{eqnarray}
x((i+1)\Delta t) &= x(i\Delta t) + v(i\Delta t) \Delta t + a(t) \frac{\Delta t^2}{2} + \mathcal{O}(\Delta t^3), \\
x((i-1)\Delta t) &= x(i\Delta t) - v(i\Delta t) \Delta t + a(t) \frac{\Delta t^2}{2} - \mathcal{O}(\Delta t^3).
\end{eqnarray}
%
Adding these two equations and rearranging terms gives us the Verlet scheme:
\begin{equation}
\boxed{x[i+1] = 2x[i] - x[i-1] + f[i]\frac{\Delta t^2}{m} + \mathcal{O}(\Delta t^4)},
\end{equation}
%
where $x[i] = x(i\Delta t)$ and $f[i] = f(x[i]) = ma[i]$. It is important to note that the Verlet scheme is accurate upto $4^{th}$ order. In addition, the Verlet scheme in the standard form does not have a dependence on the velocity field. This dependence on the velocity field can be introduced through some basic manipulation. The steps leading up to the Verlet leap-frog is derived here. To do this we start from the position update equation for the leap-frog scheme. 

\begin{equation}
x[i+1] = x[i] + w[i+1]\Delta t \implies w[i+1] = \frac{x[i+1] - x[i]}{\Delta t}
\end{equation}

Similarly, 
\begin{equation*}
w[i] = \frac{x[i]-x[i-1]}{\Delta t}.
\end{equation*}

Inserting expressions for $w[i+1]$ and $w[i]$ into the equation for update of $w$:
\begin{equation}
w[i+1] = w[i] + f[i]\frac{\Delta t}{m},
\end{equation}

gives (with rearrangements):
\begin{equation}
x[i+1] = 2x[i] - x[i-1] + f[i]\frac{\Delta t^2}{m}. 
\end{equation}

Hence, the above equation is the same as the equation for the Verlet scheme. Therefore, the Verlet leap-frog scheme and the Verlet scheme are identical. The local truncation error of the leap-frog method is order 3. However, the global error is order 2. 

\section*{Question 10}
The array returned by the Matlab function \texttt{ode45} is not equally spaced in time. Hence, even the positions are not equally spaced in time. The $\Delta t$ used by the algorithm at every time step of the simulation is plotted in Fig. \ref{fig:ode45_timestep}.

\begin{figure}[!htb]
\centering
\input{../plots/q10_ode45_timestep.tex}
\caption{The $\Delta t$ used by the \texttt{ode45} Matlab routine at different $t$ is plotted. The oscillator is subjected only to the spring force. The parameters used for this plot are: $m=2\ \mathrm{kg}$, $k=5\ \mathrm{N/m}$.}
\label{fig:ode45_timestep}
\end{figure}

As can be seen from the figure the $\Delta t$ at a given time $t$ is not constant, but rather increases by 3 orders of magnitude. The initial $\Delta t$ is maintained to be very small but later the dynamics is "fast-forwarded" by using larger $\Delta t$'s. 

\section*{Question 11}
To compare the 3 schemes - Euler, leap-frog and Runge-Kutta 4th order for the harmonic oscillator, we plot the energies for the three schemes. This is plotted in Fig. \ref{fig:energy_vs_time_integration_schemes_compare}.

\begin{figure}[!htb]
\centering
\input{../plots/q11_compare_schemes_energy_vs_time.tex}
\caption{The energy vs time for the three schemes mentioned in \textbf{Question 11} are plotted for a time step of $dt=2e-3$. The parameters used for this plot are: $m=2\ \mathrm{kg}$, $k=5\ \mathrm{N/m}$.}
\label{fig:energy_vs_time_integration_schemes_compare}
\end{figure} 

As can be observed from the figure the energy of the oscillator increases for the Euler scheme. The increase in the energy can be proven to be exponential. As for the \texttt{ode45} the energy decreases continuously. However, the energy is conserved for the Verlet leap-frog scheme for long times. 

Therefore, the Euler scheme clearly under-performs and should be avoided for systems which strictly require energy conservation. To conserve energy for long times the leap-frog algorithm performs optimally. But, the leap-frog scheme is only $2^{nd}$ order accurate. The RK4 scheme as offered by the \texttt{ode45} routine in Matlab is fourth order accurate but not energy conserving. Hence, if the aim of the study is to observe the system on short times and obtain accurate results, it would be recommended to use the RK4 scheme.   

\section*{Question 12}
\subsection*{RK4 - \texttt{ode45}}
The friction can be formulated into a matrix form acceptable to the \texttt{ode45} algorithm. The following is the formulation:

\begin{eqnarray}
\frac{dx}{dt}=v, \\
\frac{dv}{dt}=\frac{f}{m} &= -\omega_{0}^2 -\frac{\gamma}{m} v.
\end{eqnarray}

This can be recast into a matrix form which is acceptable to the \texttt{ode45} algorithm. The form of the equation after formulating the above equations into a matrix is given by:
%
\begin{equation}
\frac{d\bm{z}}{dt} = \bm{A}\bm{z},
\end{equation}
%
where,
%
\begin{equation}
\bm{z} = 
\begin{bmatrix}
x \\
v 
\end{bmatrix}
\end{equation}
%
\begin{equation}
\bm{A} = 
\begin{bmatrix}
0 & 1 \\
-\omega_{0}^2 & -(\gamma/m),
\end{bmatrix}
\end{equation}
%
where $\gamma$ is the friction coefficient. 

\subsection*{Verlet leap-frog}
The presence of friction is incorporated into the scheme by accounting for the fact that the force is velocity dependent instead of being just position dependent. The modification for the scheme is as follows:
%
\begin{equation*}
w[i+1] = w[i] + f[i]\frac{\Delta t}{m},
\end{equation*}
%
now, the force $f[i]$ is both position and velocity dependent. Substituting for the force gives us:
%
\begin{equation*}
w[i+1] = w[i] + \frac{\Delta t}{m}( -m\omega_{0}^2 x[i] - \gamma v[i] ).
\end{equation*} 
%
The velocity $v[i]$ is the velocity at the integral time step and is given by $v[i] = \frac{1}{2} ( w[i] + w[i+1] )$.\footnote{This is equivalent to $v(t) = \frac{1}{2} ( v(t + \frac{\Delta t}{2} ) + v(t - \frac{\Delta t}{2} ) )$ } Substituting for $v[i]$ and re-arranging terms finally gives us:
%
\begin{equation}
w[i+1] = a_{1}w[i] + a_{2},
\end{equation}
%
where,
%
\begin{eqnarray}
a_{1} &= \frac{1 - \frac{\gamma \Delta t}{2m}}{1 + \frac{\gamma \Delta t}{2m} }, \\
a_{2} &= -\frac{\omega_{0}^2 x[i] \Delta t}{1 + \frac{\gamma \Delta t}{2m}}.
\end{eqnarray}
%
The position is updated in the normal way, i.e. using the value of $w[i+1]$. It is given by:
\begin{equation}
x[i+1] = x[i] + w[i+1]\Delta t.
\end{equation}

We now asses the performance of both the schemes. 

\subsection*{Performance}
Analytical solution for the position and velocity can be obtained for the oscillator subjected to friction. Hence, the numerics can be checked against the the analytical solution. The position and the energy are compared in Fig. \ref{fig:ode45_vs_VerletLF_friction}. 

\begin{figure}[!htb]
\centering
\input{../plots/q12_ode45_vs_verletLF_position_vs_time.tex}
\input{../plots/q12_ode45_vs_verletLF_energy_vs_time.tex}
\caption{The position $x(t)$ and energy $E(t)$ is plotted against time for two different schemes: (1) RK4, (2) Verlet leap-frog. The two schemes are compared against the analytical solution for the oscillator in the presence of friction. The parameters used for this simulation are:$m=2\ \mathrm{kg}$, $k=5\ \mathrm{N/m}, \gamma=0.2\ \mathrm{kg/s}$.}
\label{fig:ode45_vs_VerletLF_friction}
\end{figure} 

The amplitude of the oscillator as a function of time continuously decreases about the mean position $x=0$. In addition the frequency of the oscillator is also altered due to the presence of friction. 
As can be seen from the figure both the schemes excellently predict position when compared against the analytical solution. Similarly, the total energy is also captured equally well by the two schemes. The energy initially has a oscillations of larger magnitude about the decaying exponential. As time progresses the amplitude about the decaying exponential decreases. 

\section*{Question 13}
As can be seen from Fig. \ref{fig:energy_interpretation} the energy continuously decreases due to the presence of friction. The analytical expression obtained for the decay in the energy can be obtained. 

\begin{figure}[!htb]
\centering
\input{../plots/q13_ode45_vs_verletLF_energy_decay.tex}
\caption{The energy $E(t)$ is plotted against time for the RK4 scheme. The analytical solution is plotted along with the decaying exponential with a characteristic time $\tau = \frac{1}{2\alpha}$. It is plotted for a total time of 20 periods. The parameters used for this simulation are:$m=2\ \mathrm{kg}$, $k=5\ \mathrm{N/m}, \gamma=0.2\ \mathrm{kg/s}$.}
\label{fig:energy_interpretation}
\end{figure} 

The expression for the energy is given by:
%
\begin{equation}
E(t) = E_{0}(x_{0}, v_{0})*g(\omega', t, \alpha)*\exp(-2\alpha t),
\end{equation}
%
where $E_{0}(x_{0}, v_{0})$ is the initial energy as a function of initial velocity and position, $g(\sin(\omega, \alpha, t)$ is a function containing the sinusoidal contributions to the energy and $\alpha = \frac{\gamma}{2m}$ where $\gamma$ is the friction coefficient. In addition, $\omega'=\sqrt{\omega_{0}^2 - \alpha^2}$ and hence is not equal to the natural frequency of the undamped harmonic oscillator. As can be seen from the equation the typical decay time $\tau = \frac{1}{2\alpha}$. This can be confirmed by plotting the exponential with this decay time. This is plotted in Fig. \ref{fig:energy_interpretation}. 

\section*{Question 14}
The addition of the driving force is only an extension of \textbf{Question 12}. It has to be noted that the frequency of forcing $\omega$ is not the same as the natural frequency of the harmonic oscillator $\omega_{0}$.
\subsection*{RK4 - \texttt{ode45}}
The friction and forcing can be formulated into a matrix form acceptable to the \texttt{ode45} algorithm. The following is the formulation:

\begin{eqnarray}
\frac{dx}{dt}=v, \\
\frac{dv}{dt}=\frac{f}{m} &= -\omega_{0}^2 -\frac{\gamma}{m} v + \frac{F}{m}\cos(\omega t).
\end{eqnarray}

This can be recast into a matrix form which is acceptable to the \texttt{ode45} algorithm. The form of the equation after formulating the above equations into a matrix is given by:
%
\begin{equation}
\frac{d\bm{z}}{dt} = \bm{A}\bm{z} + \bm{B},
\end{equation}
%
where,
%
\begin{equation}
\bm{z} = 
\begin{bmatrix}
x \\
v 
\end{bmatrix}
\end{equation}
%
\begin{equation}
\bm{A} = 
\begin{bmatrix}
0 & 1 \\
-\omega^2 & -(\gamma/m)
\end{bmatrix}
\end{equation}
%
\begin{equation}
\bm{B} = 
\begin{bmatrix}
0 \\
F/m
\end{bmatrix}
\end{equation}
where $\gamma$ is the friction coefficient and $F$ is the amplitude of the forcing. 

\subsection*{Verlet leap-frog}
The presence of forcing is incorporated into the scheme by adding it as an additional constant in the already present Verlet leap-frog scheme. The modification for the scheme is as follows:
%
\begin{equation*}
w[i+1] = w[i] + f[i]\frac{\Delta t}{m},
\end{equation*}
%
now, the force $f[i]$ is both position and velocity dependent. Substituting for the force gives us:
%
\begin{equation*}
w[i+1] = w[i] + \frac{\Delta t}{m}( -m\omega^2 x[i] - \gamma v[i] + F\cos(\omega t) ).
\end{equation*} 
%
The velocity $v[i]$ is the velocity at the integral time step and is given by $v[i] = \frac{1}{2} ( w[i] + w[i+1] )$.\footnote{This is equivalent to $v(t) = \frac{1}{2} ( v(t + \frac{\Delta t}{2} ) + v(t - \frac{\Delta t}{2} ) )$ } Substituting for $v[i]$ and re-arranging terms finally gives us:
%
\begin{equation}
w[i+1] = a_{1}w[i] + a_{2},
\end{equation}
%
where,
%
\begin{eqnarray}
a_{1} =& 1, \\
a_{2} =& \frac{\Delta t}{m}( -m\omega^2 x[i] + F\cos(\omega t) ).
\end{eqnarray}
%
The position is updated in the normal way, i.e. using the value of $w[i+1]$. It is given by:
\begin{equation}
x[i+1] = x[i] + w[i+1]\Delta t.
\end{equation}

\section*{Question 15}
The position vs time for an oscillator subjected to forcing and friction is plotted in Fig. \ref{fig:position_vs_time_friction_forcing}. As can be observed in Fig. \ref{fig:position_vs_time_friction_forcing}, the transient behavior is characterized by a decaying amplitude settling towards a steady state long time behavior. 

\begin{figure}[!htb]
\centering
\input{../plots/q15_verletLF_position_vs_time.tex}
\input{../plots/q15_verletLF_position_vs_time_transient.tex}
\caption{The position $x(t)$ is plotted against time for the Verlet leap-frog scheme with forcing and friction present. The solution from the scheme is compared against the analytical solution. The parameters used for this simulation are:$m=2\ \mathrm{kg}$, $k=5\ \mathrm{N/m}, \gamma=0.2\ \mathrm{kg/s}$, $F=1\ \mathrm{kg m/s^2}$. The figure below is a magnification of the above plot till the value $t=40$ (represented by the blue line).}
\label{fig:position_vs_time_friction_forcing}
\end{figure} 

As can be seen in the figure the position signal $x(t)$ contains two different frequencies. One of the frequency corresponds to the natural frequency. This natural frequency dampens out completely. On the other hand, the frequency due to the external forcing survives, as expected, and equals the frequency of the steady state solution. However, the amplitude of the steady state motion is dependent on the value of the friction co-efficient $\gamma$ present in the system. 

Now, when the system is simulated for several values of $F$ and $\omega$ we can observe that the amplitude of the steady state signal increases as the value of $F$ increases, given that the value of the $\gamma$ and $\omega$ are held constant. However, as the value of $F$ and $\gamma$ are both varied we observe a rich behavior of $A$: the steady state amplitude of the oscillator (response). 

\begin{figure}[!htb]
\begin{subfigure}{.5\textwidth}
  \centering
  \input{../plots/f_omg_contour_gamma_vary.tex}
\end{subfigure}%
\begin{subfigure}{.5\textwidth}
  \centering
  \input{../plots/q15_max_omg_ratio_vs_F_gamma_compare.tex}
\end{subfigure}
\caption{The 3D plot (left) of the final amplitude (response) to the driving force (frequency and amplitude) for  $\gamma=0.5, 0.8\ \mathrm{kg/s}$ is plotted here. On the $x$-axis is the ratio of frequency of the driving force to that of the natural frequency of the oscillator: $\omega/\omega_{0}$ and the $y$-axis is the amplitude of the external forcing. The $z$-axis is the steady state amplitude of the harmonic oscillator. The plot of the value of $\omega$ at which the maximum in $A$ occurs for a given F ( $\omega_{max}$ ) is plotted against for two different values of $\gamma=0.5, 0.8\ \mathrm{kg/s}$. }
\label{fig:q15_A_vs_omg_F}
\end{figure}

As can be seen in Fig. \ref{fig:q15_A_vs_omg_F}, for a fixed value of external forcing $F$ varying the frequency of the external forcing $\omega$ leads to an increase in the steady state amplitude $A$ until a given value $\omega = \omega_{max}$. On increasing $\omega$ further the maximum amplitude achieved then continuously decreases to a value close to 0. The interesting point to observe is that at the natural frequency $\omega_{0}$ the steady state amplitude does not diverge to $\inf$, as is the case for the undamped oscillator. Further, it is also interesting to note that the maximum amplitude $A$ does not occur at $\omega=\omega_{0}$, rather it is slightly shifted. This is illustrated in right-side plot in Fig. \ref{fig:q15_A_vs_omg_F}. For higher values of the friction co-efficient $\gamma$ the value the frequency $\omega_{max}$ at which $A$ is maximum for a given forcing $F$ is shifted away from the natural frequency. 

\section*{Advanced Question a}
Now, the values of $A/f$ can be obtained from theory and can be compared against the numerical predictions. This is illustrated in Fig. \ref{fig:AdvA_log_A_f_vs_log_omg}. This is the same representation as Fig. \ref{fig:q15_A_vs_omg_F}. The analytical expression for the steady state oscillation can be derived using the theory of $2^{\mathrm{nd}}$ order ODE's.  

\begin{figure}[!htb]
\centering
\input{../plots/log_A_f_vs_log_omg.tex}
\caption{The amplitude $A$ normalized by $f$ is plotted against the driving frequency $\omega$ in a log-log domain. The dots represent the numerical values and the lines represent the values from theory.}
\label{fig:AdvA_log_A_f_vs_log_omg}
\end{figure} 
%
Here only the expression is provided: 
\begin{equation}
\lim_{t \to \infty} x(t) = A\cos(\omega t - \phi),
\end{equation}
%
where,
\begin{eqnarray}
A =& \frac{F}{m}\frac{1}{\sqrt{ ( \omega_{0}^2 - \omega^2  )^2 +  ( \gamma\omega/m )^2 }}, \\
\phi =& \arctan( \frac{\gamma}{m}\frac{\omega}{\omega_{0}^2 - \omega^2} )
\end{eqnarray}

\section*{Advanced Question b}
The correctness of the three-dimensional motion of two balls connected by a spring can be checked using a few more tests in addition to the conservation of energy. The following are the other conserved quantities:
\begin{itemize}
\item \textbf{Linear Momentum}: Since the system of two particles are not acted upon by an external force, the total momentum of the system has to be conserved. It is instructive to look at the total momentum in the center of mass frame. Hence, the total momentum in the center of mass frame is given by:
\begin{eqnarray}
\bm{p}_{tot, com} =& \bm{p}_{A, com} + \bm{p}_{B, com}, \\
\bm{p}_{tot, com} =& m_{A}\bm{v}_{A, com} + m_{B}\bm{v}_{B, com},
\end{eqnarray}
where $v_{A, com}$ is the velocity of particle $A$ w.r.t to the $v_{com}$ \footnote{$v_{com} = \frac{m_{A}\bm{v}_{A} + m_{B}\bm{v}_{B}}{m_{A}+m_{B}}$}. On further simplification we find that the total momentum in the center of mass frame is $\bm{0}$. Since the there is no external force the total momentum in the center of mass frame equals $\bm{0}$.  
\item \textbf{Angular Momentum}: From Newton's second law we know that in the absence of any external torque the system conserves it's angular momentum. Hence, the total angular momentum of the system with respect to the center of mass should be constant. The equation is given by:
\begin{eqnarray}
\bm{L}_{tot,com} =& L_{A,com} + L_{B,com},\\
\bm{L}_{tot,com} =& r_{A,com}\times p_{A,com} + r_{B,com}\times p_{B,com},
\end{eqnarray}
where the subscript $com$ refers to the qunatity with respect to the relevant center of mass quantity \footnote{For example $r_{A,com} = r_{A} - r_{com}$ refers to the position of particle A with respect to the center of mass of the two particle system}. Now, on further simplification we have:
\begin{eqnarray}
\bm{L}_{tot,com} = 2 \mu^2 (\bm{r}_{BA} \times \bm{v}_{BA}),
\end{eqnarray}
where $\bm{r}_{BA} = \bm{r}_{B} - \bm{r}_{A}$ and the same convention for the velocity. 
\end{itemize}

Hence, the three quantities which have to be conserved are the energy (com frame), linear momentum (com frame) and the angular momentum (com frame). Now we plot these quantities to verify that they are conserved and hence the simulations are correct. 

\begin{figure}[!htb]
\begin{subfigure}{.5\textwidth}
  \centering
  \input{../plots/Adv_b_com_energy.tex}
\end{subfigure}%
\begin{subfigure}{.5\textwidth}
  \centering
  \input{../plots/Adv_b_com_ang_mom.tex}
\end{subfigure}%
\quad
\begin{subfigure}{.5\textwidth}
  \centering
  \input{../plots/Adv_b_com_lin_mom.tex}
\end{subfigure}%
\caption{The energy, angular momentum and the linear momentum in the center of mass frame is plotted for the two particle-spring system. The parameters for the simulation are: $m_{A}=m_{B}=2\ \mathrm{kg}$, $k = 5\ \mathrm{N/m}$, $l=1\ m$ with initial conditions $\bm{r}_{A}=(0, 0, 0)\ \mathrm{m}$, $\bm{r}_{B}=(1, 1, 1)\ \mathrm{m}$ and $\bm{v}_{A}=(+0.1, +0.1, +0.1)\ \mathrm{m/s}$, $\bm{v}_{B}=(-0.1, -0.1, -0.1)\ \mathrm{m/s}$.}
\label{fig:Adv_b}
\end{figure}

As can be seen from Fig. \ref{fig:Adv_b} the energy, angular momentum and the linear momentum throughout the simulation are conserved. Since the system was initialized with opposite but symmetrical velocities the linear and the angular momentum are $\bm{0}$. The Verlet scheme is used to integrate the equations of motion. 

\end{document}