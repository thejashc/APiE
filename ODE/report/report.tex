\documentclass[10pt]{article}
\usepackage{amsmath}

\begin{document}
\section*{Question 1}
The first order differential equation for the position $x(t)$ and velocity $v(t)$ are given by the following equations:
\begin{align}
\frac{dx(t)}{dt} &= v(t), \\
\frac{dv(t)}{dt} &= \frac{F(x(t))}{m}.
\end{align}
%
Now, according to the \textbf{forward Euler} method the L.H.S of the above equations can be expressed  as:
\begin{align*}
\frac{x(t + \Delta t) - x(t)}{\Delta t} &= v(t), \\
\frac{v(t + \Delta t) - v(t)}{\Delta t} &= \frac{F(x(t))}{m}.
\end{align*}
%
Using the notation $t=i\Delta t$ where $\Delta t$ is the length of the interval between two consecutive time steps, and, $i$ is an integer $1\leq i \leq \mathrm{Maxstap}$. Substituting for $t$ gives the following expression:
%
\begin{align*}
x[i+1] &= x[i] + v[i]*\Delta t, \\
v[i+1] &= v[i] + \frac{F(x[i])}{m},
\end{align*}
%
where $x(t+\Delta t) = x((i+1)\Delta t)$ which is written as $x[i+1]$ in the array form. $x(i\Delta t)$ in actual time would mean $x[i]$ in the array form.

\section*{Question 2}
For the case of a simple harmonic oscillator we have two parameters: spring constant $k [kg/s^2]$ and the mass of the block/particle attached to the oscillator $m [kg]$. Therefore, the natural frequency $\omega_{0}$ and time period $t_{p}$ of the spring is given by:

\begin{align}
\omega_{0} &= \sqrt{\frac{k}{m}},\\
t_{p} &= \frac{2\pi}{\omega_{0}}=2\pi \sqrt{\frac{m}{k}}.
\end{align}

Therefore, in our simulation, we would like to observe the oscillator over several periods, and, within each period, we would like to resolve the motion with a reasonable resolution. Hence, based on this argument, we can use the following estimates:

\begin{align*}
T &\approx \mathcal{O}(t_{p}) \\
\Delta t &\approx \mathcal{O}(t_{p}/100) \\
\end{align*}

Obviously, it is beneficial if we can find a higher $\Delta t$. This would imply a reduction in the computational time, but, we might incur errors. Hence, a right balance needs to be achieved. 

\section*{Question 3}
\paragraph{Energy conservation}: The harmonic oscillator is a system with only conservative forces. Hence, the system \textbf{should} conserve energy in time. Therefore, one of the criterion to check the correctness of the simulation is to check if the total energy at any point of time is equal to the energy at time $t=0$. The total energy $E$ as a function of time $t$ is given by:
%
\begin{align*}
E_{tot}(t) &= E_{\mathrm{kin}} + E_{\mathrm{kin}},\\
           &= \frac{1}{2}mv(t)^2 + \frac{1}{2}k(x(t)-x_{0})^2,
\end{align*}
%
where $v(t)$ is the instantaneous velocity, $x(t)$ is the instantaneous position and $x_{0}$ is the position at which the mass experiences 0 force.
 
\paragraph{Analytical solution}: Additionally, an analytical solution for the position and velocity can be derived for this system. Therefore, obtaining the position and velocity as a function of time and comparing it to the analytical expressions can help us check the correctness of the simulation. The analytical solution for the position and velocity is given by the following expressions:
%
\begin{align}
x(t) &= \frac{v_{0}}{\omega_{0}}\sin(\omega_{0}t) + x_{0}\cos(\omega_{0}t),\\
v(t) &= v_{0}\cos(\omega_{0}t) - x_{0}\omega\sin(\omega_{0}t),
\label{eqn:harmonic_oscillator_analytical_soln}
\end{align} 
%
where $x_{0}, v_{0}$ are the initial conditions for the harmonic oscillator. Hence, the above equations represent the general solution for the position and velocity, respectively. 

\paragraph{Phase space}: Finally, another check is that to obtain the trajectory of the harmonic oscillator in the phase space. For a harmonic oscillator, without any damping or forcing, the trajectory in phase space should follow a ellipse.\footnote{ This is just another manifestation of the energy conservation principle.}. The equation of the ellipse obtained from Eqn. \eqref{eqn:harmonic_oscillator_analytical_soln} after squaring, adding and rearranging the resulting terms is:

\begin{equation}
\frac{x^2(t)}{a^2} + \frac{v^2(t)}{b^2} = 1,
\end{equation}

where,
\begin{align*}
a^2 =& \Big(\frac{v_{0}^2 + x_{0}^2 \omega_{0}^2}{\omega_{0}^2}\Big),\\
b^2 =& v_{0}^2 + x_{0}^2 \omega_{0}^2.
\end{align*}

\section*{Question 4}
The oscillator has been simulated in MATLAB.

\section*{Question 5}


\section*{Question 6}

\section*{Question 7}

\section*{Question 8}

\section*{Question 9}

\section*{Question 10}

\section*{Question 11}

\section*{Question 12}

\section*{Question 13}

\section*{Question 14}

\section*{Question 15}

\section*{Advanced Question a}

\section*{Advanced Question b}
\end{document}